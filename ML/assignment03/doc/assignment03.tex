\documentclass{article}[11pt]

\usepackage{amsmath}
\usepackage{amsfonts}

\usepackage{graphicx}
\graphicspath{{../fig/}} 								% Path to a folder where all pictures are located

\usepackage{fullpage}

\title{Machine Learning -- Assignment 3}
\author{Weipeng He \\ \texttt{2he@informatik.uni-hamburg.de} \\ \texttt{6411529}}

\begin{document}

\maketitle

\section{}
\subsection{}
\texttt{genLinData.m} generates random linear data with noise. \texttt{X} is a set of random numbers that is uniformly distributed on $[-2, 1]$. \texttt{Y} is a set of numbers that obeys $Y = -0.7X + 1 + noise$, where the noise is normally distributed with variance of \texttt{sigma};

\begin{center}
  \includegraphics[width=.7\textwidth]{ex01_1};
\end{center}

\subsection{}
Please find the implementation in file \texttt{ex01.m}.

\subsection{}
The fitted line is shown below.

\begin{center}
  \includegraphics[width=.7\textwidth]{ex01_2};
\end{center}

\subsection{}
With test data.

\begin{center}
  \includegraphics[width=.7\textwidth]{ex01_3};
\end{center}

\subsection{}
The average error of different sigma is shown below.

\begin{center}
  \includegraphics[width=.7\textwidth]{ex01_4};
\end{center}

\subsection{}
With the extreme outlier the error increases saliently.

\begin{center}
  \includegraphics[width=.7\textwidth]{ex01_5};
\end{center}

\section{}
\subsection{}
Using Linear Least Square Regression, the average error is $0.975$.

\begin{center}
  \includegraphics[width=.7\textwidth]{ex02_1};
\end{center}

\subsection{}
Using Linear Least Square Regression with polynomial basis functions, the average error is $0.330$

\begin{center}
  \includegraphics[width=.7\textwidth]{ex02_2};
\end{center}

\subsection{}
The class of functions that we can learn with these basis function is:
\[ f(x) = \sum_{i=1}^{15} w_i x^i, w_i \in \mathbb{R} \]
namely the polynomial functions without zero degree coefficient and the degree of which is not more than 15.

\subsection{}
The predicted lines of Ridge Regression are shown below:

\begin{center}
  \includegraphics[width=.7\textwidth]{ex02_3};
\end{center}

\subsection{}
Prediction error with different lambda are shown below:

\begin{center}
  \includegraphics[width=.7\textwidth]{ex02_4};
\end{center}

\section{}
Suppose the training data are $n$ vectors in $d$-dimensional space.

The time complexity of predicting one sample using LLS method is $\mathcal{O}(d)$.
While, the time complexity of predicting one sample using kNN regression is $\mathcal{O}(nd)$.

The space complexity of predicting one sample using LLS method is $\mathcal{O}(d)$.
While, the space complexity of predicting one sample using kNN regression is $\mathcal{O}(nd)$.

\section{}
First, $f(x) = x^2, x\in \mathbb{R}$ is convex. Because
\begin{align*}
  tx^2 + (1-t)y^2 - (tx + (1-t)y)^2 &= tx^2 - t^2x^2 + (1-t)y^2 - (1-t)^2y^2 -2t(1-t)xy &\\
                                    &= t(1-t)(x-y)^2 &\\
                                    &\ge 0 & \forall x,y \in \mathbb{R}, t \in [0,1]
\end{align*}

Therefore
\begin{align*}
 || Y - \mathbf{X}(t\mathbf{w_1}+(1-t)\mathbf{w_2}) ||^2 &= || t(Y - \mathbf{X}\mathbf{w_1}) + (1-t)(Y-\mathbf{X}\mathbf{w_2}) ||^2 &\\
                              &\le t ||Y - \mathbf{X}\mathbf{w_2}||^2 + (1-t) ||Y - \mathbf{X}\mathbf{w_2}||^2 & \forall \mathbf{w_1},\mathbf{w_2} \in \mathbb{R}^d, t \in [0,1]
\end{align*}

Thus, $||Y-\mathbf{X}\mathbf{w}||^2$ is convex. $\square$

\end{document}
